\section{Backend}
Backend našeho projektu slouží primárně k shromažďování a následnému zpracovávání dat. Naše API dále umožňuje bezplatný přístup k těmto datům.
Backend byl vyvinut v Pythonu pomocí knihovny FastApi. Pro ukládání dat byla použita databáze PostgreSQL.

Backend se dá pomyslně rozdělit do tří částí, kde každá část má na starost: 
\begin{itemize}
  \item zpřístupnění dat pro Frontend/Developery (pomocí API) 
  \item zajištění komunikace mezi stanicí a databází 
  \item spravování objednávek stanice, prostřednictvím webové stránky (více popsáno c kapitole \ref{frontend})  
\end{itemize}

O zpřístupnění naměřených dat se starají veřejné endpointy, které může kdokoli bezplatně použít, ať už pomocí naší webové stránky, či pomocí našeho API. 
V případě použití těchto endpointů, není potřeba žádná authentifikace a uživateli vrátí vyžádaná data ve formátu JSON. Mezi tyto veřejné endpointy patří:
\begin{itemize}
  \item GET /stations
  \item GET /now/{gps} 
  \item GET /stats/{gps}
\end{itemize}

První endpoint (/stations) jednoduše vrádí GPS souřadnice všech stanice, které jsou v ten moment zapojené.
\begin{lstlisting}[language=json,firstnumber=1, caption=Příklad požadavku /stations]
{
  "message": "ok",
  "stations": [
    {
      "gps": "50.0993194\_14.3596525"
    },
    {
      "gps": "49.7454400\_14.0578025"
    }
  ]
}
\end{lstlisting}
Druhý endpoint (/now/{gps}) bere jako parametr GPS souřadnice stanice a podle nich z databáze vytáhne poslední naměřené data dané stanice.
\begin{lstlisting}[language=json,firstnumber=1, caption=Příklad požadavku /now/{gps} ]
{
  "message": "ok",
  "time": "24-12-2022 4:20:00",
  "temperature": -3,
  "humidity": 43,
  "pressure": 100000,
  "wind\_speed": 13,
  "wind\_direction": "N",
  "rain": 4
}
\end{lstlisting}
Poslední endpoint (/stats/{gps}) má za úkol vracet dlouhodobá zprůměrovaná data v časovém rozmezí, které je určeno pomocí query parametrů,
to znamená "date\_from", "date\_to" a nepovinný parametry "freq", který pevně udává frekvenci průměrování dat. Data se průměrují vzhledem k velikosti časového intervalu.
Data v časovém rozmezí, které je menší než jeden den se průměrují po hodině.
Data v časovém rozmezí, které je větší než jeden den se průměrují po dnech. Data v rozmezí, které je větší než 3 měsíce se průměrují po týdnech.
Nebo se data průměrují podle hodnoty parametru query "freq". V případě, že se freq rovná:
\begin{itemize}
  \item 1 -> data se průměrují po 5 minutách
  \item 2 -> data se průměrují po 30 minutách
  \item 3 -> data se průměrují po hodině 
  \item 4 -> data se průměrují po dnech 
  \item 5 -> data se průměrují po týdnech 
  \item 6 -> data se průměrují po dvou týdnech 
  \item 7 -> data se průměrují po čtyřech týdnech 
\end{itemize}
\begin{lstlisting}[language=json,firstnumber=1, caption=Příklad požadavku /stats/{gps}?date\_from=1-12-2022&date\_to=3-12-2022&freq=4 ]
{
  "message": "ok",
  "data": [
    {
      "time": "1-12-2022 00:00:00",
      "temperature": -3.6,
      "humidity": 41,
      "pressure": 100200,
      "wind\_speed": 13,
      "wind\_direction": "N",
      "rain": 4,
      "avrage\_of": 288
    },
    {
      "time": "2-12-2022 00:00:00",
      "temperature": -2.2,
      "humidity": 49,
      "pressure": 100100,
      "wind\_speed": 4,
      "wind\_direction": "S",
      "rain": 0,
      "avrage\_of": 288
    },
    {
      "time": "3-12-2022 00:00:00",
      "temperature": -1.7,
      "humidity": 74,
      "pressure": 100800,
      "wind\_speed": 9,
      "wind\_direction": "E",
      "rain": 9,
      "avrage\_of": 288
    }
  ]
}
\end{lstlisting}
\subsection{Databáze}

\subsection{Platební brána}
Lorem ipsum dolor sit amet, qui minim labore adipisicing minim sint cillum sint consectetur cupidatat.
