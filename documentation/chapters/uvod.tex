\section{Úvod}
Cílem tohoto projektu bylo vytvořit měřící stanici na měření aktualního počasí a následné zobrazení dat pomocí grafů na webové stránce.

\subsection{Výběr projektu}
Tento téma jsme si vybrali, abychom získali zkušenosti s vytvářením projektu, 
který obsahuje propojení hardwaru se sofwarem, konkrétně propojení stanice s backendem a následné zobrazení na webové stránce. 

Prvoplánově jsme zamýšleli vytvořit web pro inzerci produktů,
ale tento nápad se nám nezdál ideální, neboť neobsahoval programátorsky zajímavé prvky a nezdál se nám příliš tvůrčí.

\subsection{Původní zadání}
Aplikace bude mit tři části
\begin{enumerate}
    \item operace a vyrobení hardwaru, který bude měřit počasí (teplota, vlhkost, rychlost a směr větru, déšť) hardware bude obsluhovat Arduino a čidla si vyrobíme sami (rychlost a směr větru, měření deště)
    \item backend - sbírá data o počasí a ukládá je do databáze, backend bude fungovat i jako API pro jine vyvojare na zjisteni aktualniho pocasi
    \item frontend webová aplikace zobrazující počasí (mapka na které budou vidět všechny stanice a bude se dát zobrazit podrobnější data o dané stanici).
\end{enumerate}